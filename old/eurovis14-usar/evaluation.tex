\section{Evaluation} \label{sec:evaluation}
We performed a qualitative evaluation of our system with three experts. One of the experts is a field leading researcher in emergency informatics, the two others are full-time employees at the Federal Agency for Technical Relief (THW) in Germany, one of them being a station officer. As the overall number of experts in the field is not high enough to perform thorough quantitative user studies, we decided to perform a quantitative, informal study. We prepared an expressive video showing the functionalities of the system navigating the preprocessed point cloud and annotating a start point and a POI. Then, an optimal path was computed between these points, which had to be modified as hazardous environments became known. The test scenario is based on the application case presented in Section~\ref{sec:cases}. Figures~\ref{fig:teaser:2}, ~\ref{fig:teaser:3}, and~\ref{fig:teaser:4} show still shots from this video. Afterwards, we asked the experts three open questions with the request to answer as elaborate as possible. The questions and their answers were:

\noindent \textbf{"Is it helpful to display the paths and does this representation provide additional information?"} One of the experts was very convinced about the usefulness of the system and suggested many other application areas close to his daily work, which would benefit from our system, such as a fire brigade trying to extinguish a fire in a building. The limited visibility in such an environment would benefit greatly from the system according to the expert. The other two experts were positive, but not as enthusiastic.

\noindent \textbf{"Would you like to use this system in addition, or as a replacement, to your current tools?"} Two of the respondents would like to use the system in their workflow and were very positive about it. One of them could not see it as an addition, as they do not currently use any computer system in their work. This expert further noted that it would be important to know how long it would take to generate these representations to perform a cost/benefit analysis. The third expert would like to try out the system in a controlled test environment before making any predictions.

\noindent \textbf{"Is it easier to analyze the scene using our rendering method [as compared to a point-based rendering]?"} Although the replies from all respondents were positive, each suggested a useful improvement to the system. One expert thought it would be useful for 2D navigation and thought that a color picture of the building would have helped. The second expert would have liked to see a multi-story building in the video to be able to detect rubble and damaged ceilings to assess the usefulness towards detecting this kind of dangerous situations. The last respondent thought it was easier to analyze the data, but wished for a higher resolution of the voxelized point cloud data (see Figure~\ref{fig:teaser:1} for a dataset with the same voxel size).

From the answers received we can draw the conclusion that all of the experts liked the system, but wanted to test it in a real-world application case to see how well it performs. This is in line with our future plans of performing a thorough scenario-based evaluation to receive more feedback to improve the system.