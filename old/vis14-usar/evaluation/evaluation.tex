\documentclass[8pt]{article}
\usepackage[landscape]{geometry}
\usepackage[cm]{fullpage}
\usepackage{longtable}

\begin{document}
\small
% phpMyAdmin LaTeX Dump
% version 3.4.10.1deb1
% http://www.phpmyadmin.net
%
% Host: localhost
% Generation Time: Mar 25, 2014 at 11:23 AM
% Server version: 5.5.35
% PHP Version: 5.3.10-1ubuntu3.9
% 
% Database: 'eurovis14-usar'
% 

\section*{Appendix A: Evaluation}

One participant's (id=17094461) answers were given in German; these replies have been translated into English.

\subsection*{Participants}

\begin{longtable}{r l}
\hline
\textbf{id} & Unique identifier used to identify participants\\
\textbf{name} & Name of participant (optional) [retracted from document] \\
\textbf{profession} & Profession of participant \\
\textbf{affiliation} & Affiliation of participant (optional) \\
\textbf{time} & Time spend during evaluation (start time $-$ end time) \\
\hline
\end{longtable}


\begin{longtable}{|c|c|l|p{8cm}|l|l|l|} 
\hline
\multicolumn{1}{|c|}{\textbf{id}} &
\multicolumn{1}{|c|}{\textbf{name}} &
\multicolumn{1}{|c|}{\textbf{profession}} &
\multicolumn{1}{|c|}{\textbf{affiliation}} &
\multicolumn{1}{|c|}{\textbf{time}} \\
\hline \hline 
\endfirsthead 
% ---- Data begin
17094461 & A [...] & Feuerwehroffizier [Fire officer] & Berufsfeuerwehr Graz [Professional firefighters, Graz] & \textit{did not finish} \\ \hline 
213368588 & B [...] & US\&R Search Mgr &  & 42 minutes, 21 seconds \\ \hline 
424081438 & C [...] & Fire Officer & Italian Fire Corps & \textit{did not finish}, see N.B. \\ \hline 
459131233 & C [...] & Fire Officer & Italian Fire Corps & 89 minutes, 28 seconds \\ \hline 
1085248572 & D [...] & Italian firefighter engineer &  & 321 minutes, 36 seconds \\ \hline 
1095077606 & E [...] & Search \& Rescue & TX-TF1 & 53 minutes, 42 seconds \\ \hline 
182607067 & F [...] & researcher & institute of mathematical machines & 20 minutes, 56 seconds \\ \hline 
945388657 & G [...] & Researcher & Royal Military Academy & \textit{did not finish} \\ \hline 
1188622652 & H [...] & Young researcher &  & 827 minutes, 52 seconds \\ \hline 
2116651686 & I [...]  & USAR Consultant & THW - German Federal Agency for Technical Relief & 82 minutes, 30 seconds \\ \hline \hline
average & & & & 57 minutes, 47 seconds \\ \hline
$\sigma$ & & & & 28 minutes, 24 seconds \\ \hline
\end{longtable}
 
\noindent \textbf{N.B.} \\
Ids 424081438 and 459131233 belong the same person based on the provided name. For purposes of averaging, the two values for this person were averaged first and the result used as a single answer. \\
Only the beginning and finishing time was recorded, so there is uncertainty if the participant paused the evaluation

\newpage

%
%
% Data: Page1
%
%
\subsection*{3D Representation}

\begin{longtable}{r p{12.5cm} l}
\hline
\textbf{immersion} & Rate the level of immersion (the feeling of being involved, presence) in the scene. & 1: ``no immersion'' -- 5: ``high immersion''\\
\textbf{knowledge} & Rate your knowledge and understanding of the structure of the building. & 1: ``no understanding'' -- 5: ``full understanding''\\
\textbf{useful} & Rate the usefulness of the 3D rendering in understanding the building as compared to the birdseye view. & 1: ``useless'' -- 5: ``useful''\\
\textbf{description} & Describe the interior elements you can identify in the room that is shown in Image 5. \\
\textbf{comments} & Optionally provide additional feedback/wishes/comments/criticisms.
You can also describe problems or issues regarding one of the tasks/questions here.\\
\hline
\end{longtable}


\begin{longtable}{|c|c|c|c|p{8cm}|p{8cm}|} 
\hline
\multicolumn{1}{|c|}{\textbf{id}} &
\multicolumn{1}{|c|}{\textbf{immersion}} &
\multicolumn{1}{|c|}{\textbf{knowledge}} &
\multicolumn{1}{|c|}{\textbf{useful}} &
\multicolumn{1}{|c|}{\textbf{description}} &
\multicolumn{1}{|c|}{\textbf{comments}} \\
\hline \hline
\endfirsthead 
% ---- Data begin
17094461 [A] & 2 & 2 & 2 & Tische, Ablagen Aktenablagen udgl [Desks, shelves, document shelves, and the like] & ziemlich unübewichtlich [fairly cluttered]\\ \hline 
213368588 [B] & 3 & 3 & 4 & desks, book shelf, doorways, alcove or closet & The simulated depth images allow for better ability to distinguish features.  \\ \hline 
424081438 [C] & 4 & 3 & 4 & I can see a gate. It is supposed to have a staircase close to the gate. I see a barrell close to the gate. To the right there is an access to another room. There is a corridor and a long desk (a lab maybe)to the right.  & Difficulty to indentify victims and to get an idea if the structure is stable/unstable \\ \hline 
459131233 [C] & 3 & 4 & 4 & I see a gate, a barrel, close to the gate I can figure the entrance of a flight of stairs. At the rear, the entrance to another room via a corridor.  &  \\ \hline 
1085248572 [D] & 3 & 3 & 3 & it seems an office room, with shelving, tables and so on & It is not easy to identify the different objects \\ \hline 
1095077606 [E] & 5 & 3 & 5 & control station to the right and some shelves at 11 and 12 o clock. looks like a chair turned over at the work stations &  \\ \hline 
182607067 [F] & 2 & 1 & 5 & wall, ceiling , floor, some furnitures, & Please try to improve you registration algorithm. The 3D map is not accurate. Try to use more colors to distinguish objects.  \\ \hline 
945388657 [G] & 3 & 2 & 5 & cupboard & much better with simulated depth \\ \hline 
1188622652 [H] & 2 & 3 & 2 & cupboard or a bookstand & the noisy data are unfiltered
the accuracy of the model is very poor but the data can be useful \\ \hline 
2116651686 [J] & 3 & 2 & 4 & shelfs
tables
sinks
barrel & no colors! Matching of camera picture with 3D scan can improve the understanding. Poor quality of the scan, due to its mechanisms can lead to misunderstandings or wrong judgement. Colors can help improve the understanding of the real situation \\ \hline \hline
average & 2.94 & 2.5 & 3.77 & & \\ \hline
$\sigma$ & 0.95015 & 0.79057 & 1.2018 & & \\ \hline
\end{longtable}

\newpage
%
%
% Data: Page2
%
%
\subsection*{Path Representation}

\begin{longtable}{r p{15cm} c}
\hline
\textbf{scenario1} & Which evacuation path would you choose in scenario I? & 1=violet; 2=blue; 3=orange\\
\textbf{scenario2} & Which evacuation path would you choose in scenario II? & 1=violet; 2=blue; 3=orange\\
\textbf{scenario3} & Which evacuation path would you choose in scenario III? & 1=violet; 2=blue; 3=orange\\
\textbf{length} & What is the length of the blue path in relation to the violet path? \\
\textbf{sacrificesafety} & When, in general, is it useful to sacrifice safety to reduce travel time along a evacuation path? \\
\textbf{comments} & Optionally provide additional feedback/wishes/comments/criticisms.
You can also describe problems or issues regarding one of the tasks/questions here. \\
\hline
\end{longtable}
 

\begin{longtable}{|c|c|c|c|p{3.5cm}|p{6.5cm}|p{5cm}|} 
\hline
\multicolumn{1}{|c|}{\textbf{id}} &
\multicolumn{1}{|c|}{\textbf{scenario1}} &
\multicolumn{1}{|c|}{\textbf{scenario2}} &
\multicolumn{1}{|c|}{\textbf{scenario3}} &
\multicolumn{1}{|c|}{\textbf{length}} &
\multicolumn{1}{|c|}{\textbf{sacrificesafety}} &
\multicolumn{1}{|c|}{\textbf{comments}} \\
\hline \hline
\endfirsthead 
% ---- Data begin
17094461 [A] & 1 & 2 & 2 & fast doppelt so lang [almost double in length] & zeitkritische Menschenrettung [time-critical rescue of human lives]  &  \\ \hline 
213368588 [B] & 1 & 3 & 2 & Approx 4 times further & Only if safety is more compromised by increasing travel time. &  \\ \hline 
424081438 [C] & 1 & 3 & 2 & roughly 3 times the violet path & when there is a risk of imminent collapse inside the building and operators have to rescue people; when I can reduce the exposure time to the hazard area if well protected. &  \\ \hline 
459131233 [C] & 1 & 3 & 2 & roughly 3 times the violet path & to save lives when there is a low risk of radiological exposure &  \\ \hline 
1095077606 [E] & 1 & 3 & 2 & approx 3 times as long & When the reward greatly exceeds the risk. &  \\ \hline 
182607067 [F] & 1 & 2 & 3 & 2x more & depends, It is difficult to justify. Safety first! & Evacuation path are to close to obstacles. You should provide paths that could avoid collisions with for example walls, furnitures. \\ \hline 
945388657 [G] & 1 & 3 & 2 & nearly 2 times larger & never likely in the process of reaching a victim

it is more likely if the team would need to ecacuate to the exit rapidly due to sudden structure instability &  \\ \hline 
1188622652 [H] & 1 & 3 & 2 & 2 times longer & if the danger increases while staying inside e.g. if building can collapse &  \\ \hline 
2116651686 [J] & 1 & 3 & 2 & two to three times more & scarify safety in order to reduce travel time is called crash rescue. This is admissible if there is a higher risk possible to occur, like collapse, radiation,.. 
or if the wounding is not severe and only a very limited number of rescuers have to rescue a very high number of victims & It is hard to judge the real way to take as structural integrity might be an issue. The blue and the orange path for example seem to pass at some regions with highly damaged structure, which might party collapse if heavy loading occur (for rescuers with a stretcher marching). \\ \hline \hline
average & & & & 2.56x & & \\ \hline
correct & 1 & 3 & 2 & 1.54x & & \\ \hline
% ---- Data end
\end{longtable}
 
 
 \newpage
%
%
% Data: Page3
%
%
\subsection*{Evacuation Path Walkthrough}

\begin{longtable}{r p{14.5cm} l}
\hline
\textbf{usefulness} & Rate the usefulness of the walkthrough in helping to understand the path. & 1: ``useless'' -- 5: ``useful''\\
\textbf{knowledge} & Rate your knowledge and understanding of the evacuation path. & 1: ``no understanding'' -- 5: `` full understanding''\\
\textbf{path1} & Which of the videos did you inspect? --- Path I – Direct Rendering \& Simulated Depth Image \\
\textbf{path2} & Which of the videos did you inspect? --- Path II – Direct Rendering \& Simulated Depth Image\\
\textbf{obstacles1} & Did you see any potential obstacles along the way? If so, when did you see them (time in the video) and why might they be troublesome? --- Path I\\
\textbf{obstacles2} & Did you see any potential obstacles along the way? If so, when did you see them (time in the video) and why might they be troublesome? --- Path II\\
\textbf{similarities} & Did you notice similar structures you could identify in both paths? If so, when did they occur (time in the video)?\\
\textbf{comments} & Optionally provide additional feedback/wishes/comments/criticisms.
You can also describe problems or issues regarding one of the tasks/questions here.\\
\hline
\end{longtable}


\begin{longtable}{|c|c|c|c|c|p{4cm}|p{3cm}|p{2.5cm}|p{2.5cm}|} 
\hline
\textbf{id} &
\multicolumn{1}{|c|}{\textbf{usefulness}} &
\multicolumn{1}{|c|}{\textbf{knowledge}} &
\multicolumn{1}{|c|}{\textbf{path1}} &
\multicolumn{1}{|c|}{\textbf{path2}} &
\multicolumn{1}{|c|}{\textbf{obstacles1}} &
\multicolumn{1}{|c|}{\textbf{obstacles2}} &
\multicolumn{1}{|c|}{\textbf{similarities}} &
\multicolumn{1}{|c|}{\textbf{comments}} \\
\hline \hline
\endfirsthead 
17094461 [A] & 3 & 3 & direct \& depth & direct \& depth &  &  &  &  \\ \hline 
459131233 [C] & 3 & 3 & direct \& depth & direct \& depth & 1,08; from 1,28 to 1,38. 
They might be obstacles to perform rescue  & 0.27; 0.41; 1.42.  & for instance, I see in Path 1 (0,08) and in Path 2 (1,25 min) cables from the bottom to the top in specific area of the building. &  \\ \hline 
1085248572 [D] & 3 & 3 & direct \& depth & direct \& depth & time:
0.13 - 0.17;
0.48 - 0.52;
0.57;
1.27 & time 
0.06 - 0.09;
0.27; 
0.36;
0.42 & NO &  \\ \hline 
1095077606 [E] & 5 & 4 & direct \& depth & direct \& depth & 14 seconds, 48 seconds, 1:07 minutes, 1:24 minutes to 1:35 minutes & 5 seconds, 29 seconds,41 seconds,1:20 min to 1:45 min &  &  \\ \hline 
182607067 [F] & 1 & 1 & direct \& depth & direct \& depth & It will be easer by adding colours. & It will be easer by adding colours. & It will be easer by adding colours. & It will be easer by adding colours.
Grey colour make me tired looking for obstacles. 
To be honest I can see everything because I am working with such data. But the cognitive load is to much. \\ \hline 
945388657 [G] & 4 & 3 & direct \& depth & direct \& depth &  &  &  &  \\ \hline 
1188622652 [H] & 2 & 2 & direct \& depth & &  &  &  &  \\ \hline 
2116651686 [J] & 4 & 3 & direct \& depth & direct \& depth & 0:01 - 0:16: heavy rubble? $\rightarrow$ structural integrity? 0:35: hole in floor to the right? $\rightarrow$ risk of collapse 0:45 - 0:59: heavy rubble? $\rightarrow$ structural integrity? 1:09: remains of furniture? $\rightarrow$ barrier 1:20: heavy rubble? $\rightarrow$ structural integrity? 1:27: parts of the ceiling? $\rightarrow$ risk of collapse & 0:06: parts of the ceiling? $\rightarrow$ risk of collapse 0:26: remains of furniture? $\rightarrow$ barrier 0:41: hole in wall? $\rightarrow$ structural integrity? 0:59: hole in floor to the left? $\rightarrow$ risk of collapse 1:21 - 1:42: heavy rubble? $\rightarrow$ structural integrity? & Video I: 0:35: hole in floor 1:09: remains of furniture 1:27: parts of the ceiling & No color information! No textures! This does significantly help to improve the understanding of the structural integrity, of missing pieces do to a bad scan and helps significantly the orientation. \\ \hline \hline
average & 3.125 & 2.75 & & & & & & \\ \hline
$\sigma$ & 1.24642 & 0.88641 & & & & & & \\ \hline
 \end{longtable}

\newpage
%
%
% Data: Page5
%
%
\subsection*{Profile Plot}
\begin{longtable}{r p{14cm} l}
\hline
\textbf{knowledge} & Rate your knowledge and understanding of the Profile Plot. & 1: ``no understanding'' -- 5: ``full understanding''\\
\textbf{numPaths} & How many different paths exist in the plot?\\
\textbf{shortest} & Which path has the shortest length?\\
\textbf{crossings} & How often does the shortest path cross the hazardous areas?\\
\textbf{differences} & Relate the characteristics of the 'orange' and the 'red' path to each other.\\
\textbf{choice} & Which path would you choose and why?\\
\textbf{comments} & Optionally provide additional feedback/wishes/comments/criticisms.
You can also describe problems or issues regarding one of the tasks/questions here.\\
\hline
\end{longtable}


\begin{longtable}{|c|c|c|c|c|p{4.5cm}|p{4.5cm}|p{4.5cm}|} 
\hline
\multicolumn{1}{|c|}{\textbf{id}} &
\multicolumn{1}{|c|}{\textbf{knowledge}} &
\multicolumn{1}{|c|}{\textbf{numPaths}} &
\multicolumn{1}{|c|}{\textbf{shortest}} &
\multicolumn{1}{|c|}{\textbf{crossings}} &
\multicolumn{1}{|c|}{\textbf{differences}} &
\multicolumn{1}{|c|}{\textbf{choice}} &
\multicolumn{1}{|c|}{\textbf{comments}} \\
\hline \hline
\endfirsthead 
17094461 [A] & 2 & 3 & blau [blue] & 2 &  & blau [blue] &  \\ \hline 
213368588 [B] & 4 & 3 & blue & 2 & Orange path is shorter but closer to the hazard area. & Red...farther away from the hazard. &  \\ \hline 
459131233 [C] & 5 & 3 & blue & 2 times & orange shorter than the red;
orange crosses the hazardous area 1 time;
red never crosses the hazardous area;
 & this is depending on the scenarios;
I would choose the one limiting my exposition to a hazardous area and a short one.
A good compromise is orange if I have to save lives; If I have not to save lives I would choose the red one. &  \\ \hline 
1085248572 [D] & 4 & 3 & blue & 2 & The red one is longer but never cross the hazardous areas. & the red path because the distance from the hazardous areas is longer and this means more safety for rescuers &  \\ \hline 
1095077606 [E] & 4 & 3 & Blue & twice & Orange path is closer to the hazard than red but shorter & Not knowing the hazard the safest route is the red path. If the hazards can be mitigated the blue is the shortest but with the most exposure. The Orange path would probably be a compromise between time to target and exposure to a hazard. &  \\ \hline 
182607067 [F] & 1 & 3 & blue & 2 & orange is shorter and safer. & blue & it is not so obvious witch path is the best. \\ \hline 
1188622652 [H] & 4 & 3 & blue & two times & orange is shorter but the red is safe (the distance to hazard is bigger) & orange - compromise between safety and the distance travelled (minimizing total exposure time) &  \\ \hline 
2116651686 [J] & 4 & 3 & blue & two times & red path is 70/50m longer than the orange path, while the orange path passes one time a hazardous area, the red path has always at least 2m distance, but comes three times close to an hazardous area & the red path with minimal risk of exposing to an unknown risk. & Risk is unknown, but how about not identified risks by the model? How about radiation? Maybe the orange path is better, because on might expose himself to a controllable hazard? Protection agains this hazard is unknown! \\ \hline \hline
average & 3.5 & & & & & & \\ \hline
$\sigma$ & 1.30931 & & & & & & \\ \hline
correct & & 3 & blue & 2 & & & \\ \hline
\end{longtable}

\newpage
%
% Data: Page6
%
\subsection*{Parallel Coordinate Plot}
\begin{longtable}{r p{15cm} l}
\hline
\textbf{knowledge} & Rate your knowledge and understanding of the Parallel Coordinates Plot. & 1: ``no understanding'' -- 5: ``full understanding''\\
\textbf{shortest} & Which path has the shortest length?\\
\textbf{safest} & Which path is the safest and why?\\
\textbf{choice1} & Given the choice between the 'yellow' and the 'red' path, which one would you choose and why? Which trade-offs are necessary?\\
\textbf{choice2} & Given the choice between the 'blue' and the 'pink' path, which one would you choose and why? Which trade-offs are necessary?\\
\textbf{choiceAll} & Which path would you choose based on this information and why?\\
\textbf{ordering} & How would you order the attributes from more important to less important?\\
\textbf{additional} & Which path or paths would you like to inspect in the 3D view? Which additional information would you hope to gain from it?\\
\textbf{comments} & Optionally provide additional feedback/wishes/comments/criticisms.
You can also describe problems or issues regarding one of the tasks/questions here.\\
\hline
\end{longtable}


\begin{longtable}{|c|c|c|p{2.25cm}|p{2.5cm}|p{2.25cm}|p{2.25cm}|p{2.25cm}|p{2.25cm}|p{2.25cm}|} 
\hline
\multicolumn{1}{|c|}{\textbf{id}} &
\multicolumn{1}{|c|}{\textbf{knowledge}} &
\multicolumn{1}{|c|}{\textbf{shortest}} &
\multicolumn{1}{|c|}{\textbf{safest}} &
\multicolumn{1}{|c|}{\textbf{choice1}} &
\multicolumn{1}{|c|}{\textbf{choice2}} &
\multicolumn{1}{|c|}{\textbf{choiceAll}} &
\multicolumn{1}{|c|}{\textbf{ordering}} &
\multicolumn{1}{|c|}{\textbf{additional}} &
\multicolumn{1}{|c|}{\textbf{comments}} \\
\hline \hline
\endfirsthead 
213368588 [B] & 2 & lime green & dark blue: shortest distance but further away from the hazard. & Red. A little shorter but a little closer average distance to the hazard. Looked at closest distance (red was further away)overall and time overall. & Pink. Shorter and further away from hazard. & Dark blue. Shortest distance with least risk. & Closeness to hazard, time of travel. & Dk blue and red. Would like to see if hazard, though close, has some shielding between path and hazard. &  \\ \hline 
1085248572 [D] & 2 & green & blue one, because has the highest minimal and average distance from hazardous areas & yellow because has an higher average distance from hazardous areas.  & Blue one has an higher average distance from hazardous areas. Is necessary a long path. & Blue one & average distance from Haz aera
minimal distance from Haz area
path length &  &  \\ \hline 
1095077606 [E] & 2 & green & Light blue because it is the farthest from the hazard & yellow because it has it has a higher average distance to the hazard. & Blue & Not sure & I have no idea what deviation refers to in this context. & No idea &  \\ \hline 
182607067 [F] & 1 & dont know & can not read from plot & can not read from plot & can not read from plot & can not read from plot & can not read from plot & can not read from plot & can not read from plot \\ \hline 
1188622652 [H] & 1 &  &  &  &  &  &  &  &  \\ \hline 
2116651686 [J] & 2 & green & blue as it always has the largest distance to any hazard & the red path seems better as the deviation of the average distance to any hazard is lower.  & The pink path seems better, even if longer. 
The distance to risks is always higher and the supporting floor, too.  This means a locally smaller ground pressure for a given weight distributed to less surface. & pink, even is it is one of the longest ones. The longer cyan one has only minimal changes in the distance to hazard, but is still significantly longer. Support area seems also better for pink. & minimal hazard distance, average hazard distance, distance deviation, average support area, support area deviation, path length & pink and light green, to compare if the light green is an interesting option and the exposure to hazards can be justified and protection can be provided. & Other representation? Bars? Percentages? Relative numbers? \\ \hline \hline
average & 1.66 & & & & && & & \\ \hline
$\sigma$ & 0.5164 & & & & & & & & \\ \hline
correct & & green & blue & & & & & & \\ \hline
 \end{longtable}


\newpage
%
% Data: Page7
%
\subsection*{Scatterplot Matrix}
\begin{longtable}{r p{15cm} l}
\hline
\textbf{knowledge} & Rate your knowledge and understanding of the Scatterplot Matrix. & 1: ``no understanding'' -- 5: ``full understanding''	\\
\textbf{shortest} & What path or paths have the shortest path length? How did you arrive at this conclusion?\\
\textbf{distance} & What path seems to be overall the robustest path with respect to the distance from the hazard areas? How did you arrive at this conclusion? \\
\textbf{choice} & Considering the Path Length and the Average Distance to Hazard, which path would you choose and why?\\
\textbf{comments} & Optionally provide additional feedback/wishes/comments/criticisms.
You can also describe problems or issues regarding one of the tasks/questions here.\\
\hline
\end{longtable}

\begin{longtable}{|c|c|p{4.5cm}|p{4.5cm}|p{4.5cm}|p{4.5cm}|} 
\hline
\multicolumn{1}{|c|}{\textbf{id}} &
\multicolumn{1}{|c|}{\textbf{knowledge}} &
\multicolumn{1}{|c|}{\textbf{shortest}} &
\multicolumn{1}{|c|}{\textbf{distance}} &
\multicolumn{1}{|c|}{\textbf{choice}} &
\multicolumn{1}{|c|}{\textbf{comments}} \\
\hline \hline 
\endfirsthead 
213368588 [B] & 1 &  &  &  &  \\ \hline 
459131233 [C] & 1 &  &  &  &  \\ \hline 
1085248572 [D] & 1 &  &  &  &  \\ \hline 
1095077606 [E] & 1 &  &  &  & I do not understand this matrix. This is more information than I would want to interpret during a SAR mission. \\ \hline 
1188622652 [H] & 1 &  &  &  &  \\ \hline 
2116651686 [J] & 2 & there is no information provided about the overall path length - no correlation of path length with path length! & The orange path if one assumes that the left is the minimum of the criteria and the right the maximum. Not clear! Taking minimal distance for examples concludes that the blue path has the shortest distance and the orange path the highest distance to the hazard. & again, orange as I consider the correlation between left to right as rising and between the lower part and the upper part of the figure.  & indicator to help understanding and decision! \\ \hline \hline
average & 1.16 & & & & \\ \hline
$\sigma$ & 0.40825 & & & & \\ \hline
correct & & group of blue & & & \\ \hline
\end{longtable}

\newpage
%
% Data: Page8
%
\subsection*{Miscellaneous}
\begin{longtable}{r p{15cm} l}
\hline
\textbf{helpful} & Is it helpful to display the paths and does this representation provide additional information?\\
\textbf{liketouse} & Would you like to use this system in addition, or as a replacement, to your current tools?\\
\textbf{birdseye} & Rate the usefulness of the birdseye overview. & 1: ``useless'' -- 5: ``useful''\\
\textbf{rendering} & Rate the usefulness of the 3D rendering. & 1: ``useless'' -- 5: ``useful''\\
\textbf{profile} & Rate the usefulness of the Profile Plot. & 1: ``useless'' -- 5: ``useful''\\
\textbf{pcp} & Rate the usefulness of the Parallel Coordinates Plot. & 1: ``useless'' -- 5: ``useful''\\
\textbf{splom} & Rate the usefulness of the Scatterplot Matrix. & 1: ``useless'' -- 5: ``useful''\\
\textbf{comments} & Please provide additional feedback/wishes/comments about the system as a whole.\\
\hline
\end{longtable}


\begin{longtable}{|c|p{3cm}|p{4cm}|c|c|c|c|c|p{7cm}|} 
\hline
\multicolumn{1}{|c|}{\textbf{id}} &
\multicolumn{1}{|c|}{\textbf{helpful}} &
\multicolumn{1}{|c|}{\textbf{liketouse}} &
\multicolumn{1}{|c|}{\textbf{birdseye}} &
\multicolumn{1}{|c|}{\textbf{rendering}} &
\multicolumn{1}{|c|}{\textbf{profile}} &
\multicolumn{1}{|c|}{\textbf{pcp}} &
\multicolumn{1}{|c|}{\textbf{splom}} &
\multicolumn{1}{|c|}{\textbf{comments}} \\
\hline \hline
\endfirsthead 
213368588 [B] & Somewhat & In addition but not replace. & 4 & 4 & 4 & 3 & 1 &  \\ \hline 
459131233 [C] & yes & yes, before a period of experimentation & 4 & 4 & 4 & 3 & 2 &  \\ \hline 
1085248572 [D] & Yes, but the GUI should be more userfriendly and understandable & It is much complicated & 3 & 3 & 2 & 2 & 2 &  \\ \hline 
1095077606 [E] & Yes it is very helpful to designate the paths. & It would be a useful tool to add to the toolbox. It would not replace any of the search tools currently in our cache. & 4 & 5 & 5 & 2 & 1 &  \\ \hline 
182607067 [F] & yes, paths are always good.  & no comment, we are working on similar functionality so we could collaborate.
 & 3 & 5 & 5 & 1 & 1 & good job! \\ \hline 
1188622652 [H] &  &  & 5 & 4 & 3 & 1 & 1 &  \\ \hline 
2116651686 [J] & Yes, it is helpful for orientation purpose and then for decision based on the scan data. Not detected hazards like structural integrity get visible by the density of scan points. & Current tools are in the USAR context either satellite pictures or bird-eye view pictures of UAVs. Therefore this is an addition which is warmly welcome! The use of UAV does however depend on the national regulations. As international USAR teams (in UN  INSARAG context) cannot be fully prepared for each single affected country, this reduces the use of even micro UAVs and UGVs & 5 & 4 & 3 & 3 & 1 & only scan data is not enough. Decision support is warmly welcome as the situation puts every rescuer under stress. However it has to be "NON-scientific", which means it has to be intuitive. Working under these circumstances does not happen every day and even while being trained on these tools, rescuers have to know also other tools and have to acquire knowledge in different areas, too. The more intuitive the decision support is, the more it is accepted and used. The worst case scenario has to be regarded, too: The best trained rescuer is not present while the system is needed.

Therefore the information has to be reduced for a normal operator. One screen has to contain all data without to many curves, graphs, … Fast decision support.
For more experienced users, additional information and data can be switched on. Therefor two modes satisfy all needs. \\ \hline \hline
average & & & 4 & 4.14 & 3.71 & 2.14 & 1.28 & \\ \hline
\end{longtable}

\end{document}
